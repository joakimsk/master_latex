\chapter{gSOAP}
\paragraph{Introduction}
gSOAP is a toolkit used for interacting with SOAP and REST web services. It takes as an input, any WSDL-file, and generates code stubs for both client and service side.

\url{http://www.cs.fsu.edu/~engelen/soap.html}

\url{http://www.drdobbs.com/cpp/gsoap-web-services/184401909}

soapdoc2.pdf is highly recommended reading and can be found inside the gSOAP package.

As an example, let us use the calculator service provided by the author of gSOAP.
\paragraph{Example}
We download the wsdl and create a header-file, describing the service.
\begin{lstlisting}[style=BashInputStyle]
    wsdl2h -o calc.h http://www.genivia.com/calc.wsdl
\end{lstlisting}

We construct the code stubs for the client in C++ and output the files to directory ccp\_files.
\begin{lstlisting}[style=BashInputStyle]
    soapcpp2 -C -d cpp_files -i calc.h
\end{lstlisting}

We then copy all files present in cpp\_files into our project folder, and reference them from the project.

We also copy the stdsoap2.h and stdsoap2.cpp files into the same project folder.

Notice that there exists several XML-documents, these describes the available functions in the service.

After which, we create a program that includes the service proxy and namespace bindings. It also calls the service function add, which accepts two doubles and returns the sum.

\lstinputlisting[language=C++, frame=single, title=\lstname, numbers=left, commentstyle=\color{mygreen}]{gsoap.cpp}